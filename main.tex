%!TEX TS-program = xelatex 
%!TEX encoding = UTF-8 Unicode
%
%  my_title
%
%  Created by David Perez on march 2010.
%  Copyright (c) David Perez 2010. All rights reserved.
% 

% -*- Mode:TeX -*-



%% The documentclass options along with the pagestyle can be used to generate
%% a technical report, a draft copy, or a regular thesis.  You may need to
%% re-specify the pagestyle after you \include  cover.tex.  For more
%% information, see the first few lines of mitthesis.cls. 

%\documentclass[12pt,vi,twoside]{mitthesis}
%%
%%  If you want your thesis copyright to you instead of MIT, use the
%%  ``vi'' option, as above.
%%
%\documentclass[12pt,twoside,leftblank]{mitthesis}
%%
%% If you want blank pages before new chapters to be labelled ``This
%% Page Intentionally Left Blank'', use the ``leftblank'' option, as
%% above. 

%For the final thesis
\documentclass[12pt,vi]{mitUNEthesis}
\pagestyle{plain} 
 
%for technical report
%\documentclass[12pt,vi,twoside]{mitthesis}
%\pagestyle{plain}

%for draft
%\documentclass[11pt,singlespace,draft]{mitthesis}
%\pagestyle{drafthead}


%%package for TODO lists, comments, annotations on pdf, with colors
\usepackage[spanish, colorinlistoftodos]{todonotes}
%%el comando de abajo hace el trabajo automaticamente identificando si es un
%%draft
%\usepackage[obeyDraft]{todonotes}
%%los posibles COLORES usados para \todo[color=green!40]{test12} son:
%%green, red, blue!20!white (para mezclar dos colores)
%%la UBICACION del comentario, en el borde o entre parrafos:
%%\todo[noline]{A note with no line ...} noline o line o inline
%%\missingfigure[figwidth=6cm]{Testing a long text string}


%%Para hacer funcionar el idioma espanhol (los acentos por ejemplo)
\usepackage[latin1]{inputenc}
\usepackage[spanish]{babel}

%este es un listing of code, creo que no voy a usar aca
%\usepackapge{lgrind}

%% in case I include .eps versions of all figures for those w/o pdftex
%\usepackage{ifpdf}
%\ifpdf
%\usepackage[pdftex]{graphicx}
%\else
\usepackage{graphicx}
%\fi

\usepackage{amsmath,amssymb,amsfonts} % easily align arrays of equations
\usepackage{hyperref} % hyperlinks
\usepackage{subfig} % multi-part figures
\usepackage{pdfpages} % makes it easy to insert all those nsigmapi fits



\begin{document}

%\include{cover}

\pagestyle{plain}
\include{contents}


\chapter{Conceptos fundamentales}

\section{Conceptos de termodin\'amica}
	
	\subsection{Definici\'on}
	Es la ciencia que se ocupa del estudio de las transformaciones 
	de energ\'ia, fundamentalmente de trabajo en calor y de calor en trabajo.
	\begin{center}
		\begin{tabular}{l l || l l}
			\textbf{Q} & calor (Kcal) & \textbf{L} & trabajo (Kgfm) \\
			\textbf{q} & calor espec\'ifico 
				\begin{math} 
					\left(\frac{Kcal}{Kg}\right)
				\end{math} 	

						& \textbf{l} & trabajo espec\'ifico
							\begin{math}
								\left(\frac{Kgfm}{Kg}\right)
							\end{math}

		\end{tabular}
		\begin{math}
			L  - - >  Q   %\todo{aqui falta un Enter}
			Q  - - >  L   
		\end{math}
	\end{center}
	
	\subsection{Importancia de la termodin\'amica}
	Termodin\'amica es la materia previa, te\'orica y fundamental para el estudio
	de las m\'aquinas t\'ermicas, compresores, m\'aquinas a combusti\'on interna 
	y externa, m\'aquinas frigor\'ificas, turbinas a gas y vapor y sistemas de 
	condicionamiento de aire.
	
	\subsection{Desenvolvimiento}
	El desenvolvimiento de la ciencia de la termodin\'amica est\'a basada en dos 
	principios fundamentales, el primer principio de la termodin\'amica o principio de
	la equivalencia o de la conservaci\'on de la energ\'ia y el segundo principio 
	de la termodin\'amica (principio de Carnot-Clausius) de caracter cualitativo.
		\subsubsection{Primer Principio de la Termodin\'amica}
		\emph{Principio de caracter cuantitativo o de equivalencia.}
		Es siempre posible transformar calor en trabajo y trabajo en calor 
		y siempre va a existir una relaci\'on constante entre esas dos grandezas, 
		si el sistema es cerrado.
			\[
				\frac{L}{Q} = Cte = X
			\]

			\[
				\frac{Q}{L} = Cte = Y 
			\]

		\subsubsection{Segundo Principio de la Termodin\'amica}
		\emph{Principio de caracter cualitativo o de Carnot-Clausius}
		Es m\'as f\'acil transformar trabajo en calor de que calor en trabajo.
			\paragraph{Observaci\'on:} 
			Calor es la degradaci\'on m\'axima de la 
			energ\'ia, calor es una forma degenerada de energ\'ia.

	\subsection{Termodin\'amica Qui\'imica}	
	Estudia las reacciones qu\'imicas desde el punto de vista del calor. Reacciones
	exot\'ermicas y endot\'ermicas.

	\subsection{Termodin\'amica T\'ecnica}
	Estudia la obtenci\'on, aprovechamiento y aplicaci\'on del trabajo.





\section{Sistema de Unidades}

	\subsection{Grandezas f\'isicas}
	Todo lo que puede ser pesado, medido y comparado.

	\subsection{Dimensiones de las grandezas f\'isicas}
	Son evaluadas por comparaci\'on, la unidad es el medio de comparaci\'on.

	\subsection{Sistema de Unidades}
	Hecho para padronizar y orientar.

	\subsection{Clasificaci\'on de los Sistemas de Unidad}
		\begin{itemize}
			\item	Sistema de unidad gravitacional o t\'ecnico - LFT.
			\item	Sistema absoluto - LMT.
				L = longitud
				F = fuerza
				M = masa
				T = tiempo
			\item	Sistema de unidades absolutas - LMT
				CGS y MKS
			\item	Sistema de unidades gravim\'etricas - LFT
				M
				Kgf
				S
		\end{itemize}
	
	




 \appendix 
% \include{appendices/appa}
%% \include{appendices/appb}
\begin{singlespace}


%esta es la bibliografia standard de la IEEE
%\bibliographystyle{IEEEtran}
%\bibliography{IEEEabrv,bibliography/myIEEEbibliography}

%\bibliography{main}
%\bibliographystyle{plain}
\end{singlespace}
\end{document}

