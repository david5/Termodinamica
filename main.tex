%!TEX TS-program = xelatex 
%!TEX encoding = UTF-8 Unicode
%
%  my_title
%
%  Created by David Perez on march 2010.
%  Copyright (c) David Perez 2010. All rights reserved.
% 

% -*- Mode:TeX -*-



%% The documentclass options along with the pagestyle can be used to generate
%% a technical report, a draft copy, or a regular thesis.  You may need to
%% re-specify the pagestyle after you \include  cover.tex.  For more
%% information, see the first few lines of mitthesis.cls. 

%\documentclass[12pt,vi,twoside]{mitthesis}
%%
%%  If you want your thesis copyright to you instead of MIT, use the
%%  ``vi'' option, as above.
%%
%\documentclass[12pt,twoside,leftblank]{mitthesis}
%%
%% If you want blank pages before new chapters to be labelled ``This
%% Page Intentionally Left Blank'', use the ``leftblank'' option, as
%% above. 

%For the final thesis
\documentclass[12pt,vi]{mitUNEthesis}
\pagestyle{plain} 
 
%for technical report
%\documentclass[12pt,vi,twoside]{mitthesis}
%\pagestyle{plain}

%for draft
%\documentclass[11pt,singlespace,draft]{mitthesis}
%\pagestyle{drafthead}


%%package for TODO lists, comments, annotations on pdf, with colors
\usepackage[spanish, colorinlistoftodos]{todonotes}
%%el comando de abajo hace el trabajo automaticamente identificando si es un
%%draft
%\usepackage[obeyDraft]{todonotes}
%%los posibles COLORES usados para \todo[color=green!40]{test12} son:
%%green, red, blue!20!white (para mezclar dos colores)
%%la UBICACION del comentario, en el borde o entre parrafos:
%%\todo[noline]{A note with no line ...} noline o line o inline
%%\missingfigure[figwidth=6cm]{Testing a long text string}


%%Para hacer funcionar el idioma espanhol (los acentos por ejemplo)
\usepackage[latin1]{inputenc}
\usepackage[spanish]{babel}

%este es un listing of code, creo que no voy a usar aca
%\usepackapge{lgrind}

%% in case I include .eps versions of all figures for those w/o pdftex
%\usepackage{ifpdf}
%\ifpdf
%\usepackage[pdftex]{graphicx}
%\else
\usepackage{graphicx}
%\fi

\usepackage{amsmath,amssymb,amsfonts} % easily align arrays of equations
\usepackage{hyperref} % hyperlinks
\usepackage{subfig} % multi-part figures
\usepackage{pdfpages} % makes it easy to insert all those nsigmapi fits



\begin{document}

%\include{cover}

\pagestyle{plain}
  % -*- Mode:TeX -*-
%% This file simply contains the commands that actually generate the table of
%% contents and lists of figures and tables.  You can omit any or all of
%% these files by simply taking out the appropriate command.  For more
%% information on these files, see appendix C.3.3 of the LaTeX manual. 
\tableofcontents
\newpage
\listoffigures
\newpage
\listoftables




\chapter{Conceptos fundamentales}

\section{Conceptos de termodin\'amica}
	
	\subsection{Definici\'on}
	Es la ciencia que se ocupa del estudio de las transformaciones de energ\'ia, fundamentalmente de trabajo en calor y de calor en trabajo.

%	Q \rightarrow calor kcal
%	q \rightarrow calor espec\'ifico (Kcal/Kg)

%	L \rightarrow trabajo (Kgfm)
%	l \rightarrow trabajo espec\'ifico (Kgfm/Kg)


%	L \rightarrow Q
%	Q \rightarrow L











 \appendix 
% \include{appendices/appa}
%% \include{appendices/appb}
\begin{singlespace}


%esta es la bibliografia standard de la IEEE
%\bibliographystyle{IEEEtran}
%\bibliography{IEEEabrv,bibliography/myIEEEbibliography}

%\bibliography{main}
%\bibliographystyle{plain}
\end{singlespace}
\end{document}

