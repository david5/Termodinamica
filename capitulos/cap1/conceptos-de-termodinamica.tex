\section{Conceptos de termodin\'amica}
	
	\subsection{Definici\'on}
	Es la ciencia que se ocupa del estudio de las transformaciones 
	de energ\'ia, fundamentalmente de trabajo en calor y de calor en trabajo.
	\begin{center}
		\begin{tabular}{l l || l l}
			\textbf{Q} & calor (Kcal) & \textbf{L} & trabajo (Kgfm) \\
			\textbf{q} & calor espec\'ifico 
				\begin{math} 
					\left(\frac{Kcal}{Kg}\right)
				\end{math} 	

						& \textbf{l} & trabajo espec\'ifico
							\begin{math}
								\left(\frac{Kgfm}{Kg}\right)
							\end{math}

		\end{tabular}
		\begin{math}
			L  - - >  Q   %\todo{aqui falta un Enter}
			Q  - - >  L   
		\end{math}
	\end{center}
	
	\subsection{Importancia de la termodin\'amica}
	Termodin\'amica es la materia previa, te\'orica y fundamental para el estudio
	de las m\'aquinas t\'ermicas, compresores, m\'aquinas a combusti\'on interna 
	y externa, m\'aquinas frigor\'ificas, turbinas a gas y vapor y sistemas de 
	condicionamiento de aire.
	
	\subsection{Desenvolvimiento}
	El desenvolvimiento de la ciencia de la termodin\'amica est\'a basada en dos 
	principios fundamentales, el primer principio de la termodin\'amica o principio de
	la equivalencia o de la conservaci\'on de la energ\'ia y el segundo principio 
	de la termodin\'amica (principio de Carnot-Clausius) de caracter cualitativo.
		\subsubsection{Primer Principio de la Termodin\'amica}
		\emph{Principio de caracter cuantitativo o de equivalencia.}
		Es siempre posible transformar calor en trabajo y trabajo en calor 
		y siempre va a existir una relaci\'on constante entre esas dos grandezas, 
		si el sistema es cerrado.
			\[
				\frac{L}{Q} = Cte = X
			\]

			\[
				\frac{Q}{L} = Cte = Y 
			\]

		\subsubsection{Segundo Principio de la Termodin\'amica}
		\emph{Principio de caracter cualitativo o de Carnot-Clausius}
		Es m\'as f\'acil transformar trabajo en calor de que calor en trabajo.
			\paragraph{Observaci\'on:} 
			Calor es la degradaci\'on m\'axima de la 
			energ\'ia, calor es una forma degenerada de energ\'ia.

	\subsection{Termodin\'amica Qui\'imica}	
	Estudia las reacciones qu\'imicas desde el punto de vista del calor. Reacciones
	exot\'ermicas y endot\'ermicas.

	\subsection{Termodin\'amica T\'ecnica}
	Estudia la obtenci\'on, aprovechamiento y aplicaci\'on del trabajo.




